\subsection*{대회 규칙}
\begin{itemize}
    \item 사용 가능한 언어는 \textbf{C}, \textbf{C++}, \textbf{Java}, \textbf{Python 3}, \textbf{PyPy 3}입니다.
    \begin{itemize}
        %\item 컴파일 옵션은 `\textit{컴파일 옵션}' 절을 참고해주세요.
        \item 모든 문제에 대해 제약 조건을 만족하며 정답을 출력하는 C++17 코드가 있음이 보장됩니다.
    \end{itemize}
    \item 대회는 대회 전용 DOMjudge 사이트에서 치뤄지며 문제, 채점 실시간 정보 등을 확인할 수 있습니다.
    \item 순위는 푼 문제가 많은 순서대로, 푼 문제 수가 같을 경우에는 패널티의 합이 낮은 순으로 정렬됩니다.
    \begin{itemize}
        \item 문제별 패널티는 `(문제를 풀기까지 걸린 시간(분)) + (그 전까지 제출한 횟수) $ \times $ 20'입니다.
        \item 컴파일 에러는 제출 횟수에 포함되지 않습니다.
    \end{itemize}
\end{itemize}
\subsection*{금지 / 제한 행위}
\begin{itemize}
    \item 대회가 진행되는 동안 화장실 등을 다녀오는 것은 자유이나, 층 별 이동은 제한됩니다.
    \item 대회 중도 퇴실은 불가합니다.
    \item 컴퓨터를 두 대 이상 사용하는 것을 금합니다.
    \item 운영진에게 질문하는 것 외에 다른 사람과 대화하는 것을 금합니다.
    \item 사전에 코드를 미리 작성해 와서 사용하는 것을 금합니다.
    \item 허용된 레퍼런스 페이지를 제외한 메신저, 인터넷 검색, 대화, 이동식 저장 매체를 통한 문제 풀이를 금합니다.
    \item 문제 제출을 비정상적으로 많이 시도하거나, 의도적으로 대회 웹 서버를 공격하는 행위를 금합니다.
\end{itemize}
대회 규칙을 어기거나, 운영진이 판단하기에 부정한 행위를 저지를 경우 경고 없이 대회 참가 자격이 박탈될 수 있습니다.
