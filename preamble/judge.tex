\subsection*{DOMjudge 채점}
DOMjudge에 코드를 업로드할 때는 다음 조건을 지켜야 합니다.
\begin{itemize}
    \item 파일 이름은 알파벳 및 숫자로 시작해야 하며, 알파벳 대소문자 / 숫자 / \verb|+._-|만 사용 가능합니다.
    \item 확장자는 C는 \verb|.c|, C++는 \verb|.cpp| / \verb|cc| /  \verb|cxx| / \verb|c++|, Java는 \verb|.java|, Python은 \verb|.py| / \verb|.py3|여야 합니다.
    \item \textbf{제출한 코드는 표준 입출력만으로 통신하여야 합니다 (파일 입출력은 금지됩니다).}
    \item 제출한 소스코드의 크기는 256 MiB 이하여야 합니다.
\end{itemize}

\subsubsection*{DOMjudge 채점 결과}
Submit을 한 다음에 Scoreboard 탭에서 제출 결과를 확인할 수 있습니다.
\begin{itemize}
    \item {\color{gray}\texttt{PENDING}} : 제출되었으며, 채점 대기중이거나 채점중입니다.
    \item {\color{my-green}\texttt{CORRECT}} : 제출한 코드가 모든 테스트 케이스에 대해 시간 제한 / 메모리 제한 내에서 올바른 답을 내었고, 정상적으로 종료되었습니다. 이 경우 제출자는 해당 문제를 \textbf{풀었습니다}.
    \item {\color{red}\texttt{COMPILER-ERROR}} : 컴파일 과정 중에 에러가 발생하여 채점이 진행되지 않았습니다.
    \item {\color{red}\texttt{TIMELIMIT}} : 프로그램 수행 시간이 제한 시간을 초과하였습니다.
    \item {\color{red}\texttt{RUN-ERROR}} : 프로그램 수행 중 에러가 발생하였습니다. (예시 : 0으로 나누기, 잘못된 주소 참조)
    \item {\color{red}\texttt{WRONG-ANSWER}} : 프로그램이 오답을 출력하였습니다.
    \item {\color{red}\texttt{OUTPUT-LIMIT}} : 프로그램이 지나치게 많은 출력을 하였습니다.
\end{itemize}
대회 중 `request clarification' 탭을 통해 주최진에게 질문을 물을 수 있습니다.
