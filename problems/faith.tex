\begin{problem}{\kcpcprobfaith\ (\kcpcprobfaithshort)}
    {표준 입력}{표준 출력}
    {\kcpcprobfaithtime\,초}{\kcpcprobfaithmemory\,MB}{}
    
    새내기 시절 안수빈은 중앙광장에서 코딩하던 도중 노트북 배터리를 모두 소모한 적이 있다. 이에 수빈이는 화를 참지 못하고 발을 세게 굴렀고, 그 때의 여파로 3층 건물 하나스퀘어는 지하에 박히게 되었다. 그 때의 여파로 노벨 광장 한복판에 박힌 석조물은 아직까지도 뽑히지 않아 지금은 랜드마크로 자리잡게 되었다.
    
    하지만 그런 비화를 모르는 수많은 사람들은 안암에 토착 신이 존재한다고 믿게 되었다. 토착 신을 믿는 신도들은 토착 신에게 기도를 드리며 신앙을 증명하며, 한편으로 하나스퀘어 사태 복구를 위한 성금을 주기적으로 낸다. 각자의 신앙심 $f_i$나 헌금 금액 $m_i$는 양의 정수로 표현된다.
    
    이들을 이끄는 사제는 신도들의 마음을 조종할 수 있다. 사제는 빨간 약과 파란 약을 이용해서 신앙심이나 내고자 하는 성금을 조작할 수 있다.
    
    \begin{itemize}
    \item 빨간 약: 먹은 사람의 신앙심의 값을 2배로 증가시킨다. ($f_i := 2 f_i$)
    \item 파란 약: 먹은 사람의 헌금 금액을 신앙심의 값으로 바꾼다. ($m_i := f_i$)
    \end{itemize}
    
    사제가 빨간 약과 파란 약을 신도들에게 적절히 먹였을 때 얻을 수 있는 헌금 총합을 최댓값을 구해라. 약을 모두 사용할 필요는 없다. 
    
    
    \InputFile
    첫째 줄에 신도의 수 $N$, 빨간 약의 수 $R$, 파란 약의 수 $B$가 공백으로 구분되어 주어진다. ($1 \le N \le 1,000$, $0 \le R\le 20$, $0 \le B \le 1,000$)
    
    이후 $N$개의 줄에 걸쳐 두 정수 $f_i$와 $m_i$가 공백으로 구분되어 주어진다. $(1 \leq f_i, m_i \leq 100,000)$ 이는 $i$번째 신도의 신앙심과 헌금 금액을 의미한다.
    
    \OutputFile
    첫 번째 줄에 사제가 빨간 약과 파란 약을 신도들에게 적절히 먹였을 때 얻을 수 있는 헌금 총합의 최댓값을 출력한다.
    
    \Examples
    \begin{example}
        \exmp{
            3 1 2
            3 5
            3 1
            2 2
        }{%
            13
        }%
        \exmp{
            5 2 2
            8 7
            5 2
            17 100
            10 12
            5 5
        }{%
            157
        }%
    \end{example}
\end{problem}

