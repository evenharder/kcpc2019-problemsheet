\begin{problem}{\kcpcprobtrap\ (\kcpcprobtrapshort)}
    {표준 입력}{표준 출력}
    {\kcpcprobtraptime\,초}{\kcpcprobtrapmemory\,MB}{}
    
    이동관은 $ N \times M$ 격자 모양의 함정에 빠졌다. 모든 격자에는 숫자가 적혀있는 트램펄린이 존재하고, 트램펄린을 통해서만 다른 격자로 이동할 수 있다.
    
    트램펄린의 이동 규칙은 다음과 같다.
    
    \begin{itemize}
        \item 함정 밖으로 이동할 수는 없다. 새로 도착하는 위치를 $(a, b)$라고 하면, $1 \leq a \leq N$과 $1 \leq b \leq M$를 만족해야 한다.
        \item 트램펄린에 $ x $가 적혀 있다면, 상하좌우로 $ x $칸 떨어진 곳으로만 이동할 수 있다. 즉, 현재 위치가 $(a, b)$라면, $(a+x, b)$, $(a-x, b)$, $(a, b+x)$, $(a, b-x)$로만 이동할 수 있다. 단, 위에 언급된 것처럼 함정 밖으로는 이동할 수 없다.
    \end{itemize}
    
    한 번 이동하는데 1만큼의 시간이 걸린다고 할 때, 탈출구까지 도달하는데 걸리는 최단 시간을 구하여라. 이동관의 출발 위치는 항상 $(1, 1)$이다.
    
    \InputFile
    첫 번째 줄에 함정의 행의 개수 $ N $과 열의 개수 $ M $이 공백으로 구분되어 주어진다. $( 1 \leq N, M \leq 1,000 ) $
    
    이후 $N$개의 줄에 걸쳐 $M$개의 자연수 $a_{1}, a_{2}, \cdots, a_{m}$이 공백으로 구분되어 주어진다. $ (1 \leq a_{j} \leq min(N,M)) $ $i+1$번째 줄의 $a_{j}$는 $ (i, j) $에 있는 트램펄린에 적혀 있는 수이다.
    
    마지막 줄에는 두 자연수 $x$, $y$가 공백으로 구분되어 주어진다. $(1 \leq x \leq N, 1 \leq y \leq M) $ 이는 탈출구가 $ (x, y) $에 존재함을 의미한다. 출발 위치가 탈출구일 수도 있다.
    \OutputFile
    첫 번째 줄에 동관이가 탈출할 수 있다면 탈출하는데 걸리는 시간을 출력한다. 
    
    만약 탈출할 수 없으면, \texttt{-1}을 출력한다.
   
    \Examples
    
    \begin{example}
        \exmp{
            3 3
            2 1 2
            1 1 1
            2 1 2
            2 2
        }{%
            -1
        }%
        \exmp{
            2 3
            1 1 1
            2 1 1
            2 3
        }{%
            2
        }%
    \end{example}
    
    \Explanation
    첫 번째 예제에서, 동관이는 $ (1,1) $, $ (1,3) $, $ (3,1) $, $ (3,3) $ 이외의 격자에는 갈 수 없기 때문에 평생 트램펄린을 탈 것이고, 따라서 탈출구까지 갈 수 없다.
    
    두 번째 예제는 $ (1,1) \rightarrow (2,1) \rightarrow (2,3)$로 간다면 2번만에 이동할 수 있다. 이보다 빠른 방법은 없다.
    
    %\Notes
    %여기는 비고 설명 (선택)
    
\end{problem}

