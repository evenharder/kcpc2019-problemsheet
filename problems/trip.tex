\begin{problem}{\kcpcprobtrip\ (\kcpcprobtripshort)}
    {표준 입력}{표준 출력}
    {\kcpcprobtriptime\,초}{\kcpcprobtripmemory\,MB}{}
    
    기묘한 세계의 이상한 도시에 사는 앨리스는 다른 도시로 여행을 떠날 계획을 세웠다. 기묘한 세계에는 총 $N$개의 도시가 있으며, $M$개의 양방향 도로로 연결되어 있다. 각 도시는 순서대로 $1$번부터 $N$번까지 번호가 붙어 있으며 앨리스가 사는 도시의 번호는 $1$번이다. 각 도로는 이용하는데 필요한 비용 $v$가 있고, 도로를 이용하기 위해서는 돈을 내야 한다. 기묘한 세계에서 도로 이용 비용을 부과하는 방법은 다음과 같다.
    
    목적지에 도달할 때 까지 이용한 도로의 비용을 순서대로 $v_{1}, v_{2}, \cdots, v_{k}$라고 했을 때 $v_{1} \oplus v_{2} \oplus \cdots \oplus v_{k}$가 부과된다. ($\oplus$는 bitwise xor이다.)
    
    앨리스는 가난하기 때문에 여행을 떠나기 전에 가장 효율적인 여행 방법을 구하려고 한다. 계산에 서툰 앨리스를 대신하여 앨리스가 사는 도시를 제외한 모든 도시까지 가는 최소 비용을 구하는 프로그램을 작성하시오.
    
    앨리스가 사는 도시의 번호는 $1$번이다. 원하는 목적지에 도달할 때까지 이용하는 도로의 개수에 제한은 없으며, $1$번 도시에서 출발해 도로들을 이용하여 다른 모든 도시에 도달할 수 있음이 보장된다.
   
    \InputFile
    첫 번째 줄에 도시의 수 $N$과 도로의 수 $M$이 공백으로 구분되어 주어진다. 
    
    ($2 \le N \le 2 \times 10^5$,  $ N-1 \le M \le 2 \times 10^5$)
    
    두 번째 줄부터 $M$개의 줄에 걸쳐 도로의 정보를 의미하는 세 정수 $a, b, c$가 공백으로 구분되어 주어진다. ($1 \le a, b \le N$, $a \neq b$, $0 \le c \le 10^9$)
    이는 도시 $a$와 도시 $b$ 사이에 $c$의 비용을 가지는 도로가 존재한다는 뜻이다.
    
    두 도시를 직접적으로 연결하는 도로는 최대 1개 있다.
    
    \OutputFile
    첫 번째 줄에 $N-1$개의 수 $C_2, C_3, \cdots, C_{N}$를 공백으로 구분하여 출력한다. $C_i$는 1번 도시에서 $i$번 도시로 가는 최소 비용이다.
    
    \Examples
    \begin{example}
        \exmp{
            3 3
            1 2 1
            2 3 5
            1 3 6
        }{%
            1 4 
        }%
        \exmp{
            4 5
            1 2 5
            2 3 7
            1 3 1
            4 1 0
            3 4 4
        }{%
            0 1 0
        }%
    \end{example}
    
\end{problem}

