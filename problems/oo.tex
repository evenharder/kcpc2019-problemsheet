\begin{problem}{\kcpcproboo\ (\kcpcprobooshort)}
    {표준 입력}{표준 출력}
    {\kcpcprobootime\,초}{\kcpcproboomemory\,MB}{}
    
    영수의 이름에 ㅇ이 2개 들어가있다는 점에서 알 수 있듯 영수는 2개의 원을 좋아한다.
    
    영수는 좌표평면에 $N$개의 점을 찍었다. 이 $N$개의 점은 적절한 2개의 원을 정해서 모든 점이 적어도 어느 하나의 원 위에 존재하게끔 할 수 있음이 보장된다. $N$개의 점이 주어질 때 모든 점이 적어도 어느 하나의 원 위에 존재하게끔 하는 2개의 원을 구해보자.
    
    \InputFile
    첫번째 줄에 점의 수 $N$이 주어진다. ($1 \le N \le 20$)
    
    둘째 줄부터 $N$개의 줄에 걸쳐 실수 $x, y$가 소숫점 아래 15자리까지 공백으로 구분되어 주어진다. $(x,y)$는 점의 좌표를 의미한다. ($ -10^6 \le x, y \le 10^6$)
    
    임의의 두 점 사이의 거리가 $1.1$ 이상임이 보장된다.
    
    출력할 때 $x, y, r$의 소숫점 아래 자리수는 제한이 없지만 C++ double로 읽어들여 답을 판정하기 때문에 16자리부터는 의미가 없다.
    
    \OutputFile
    두 줄에 걸쳐 원의 중심의 좌표 $x, y$와 반지름 $r$을 공백으로 구분하여 출력한다. 이 때 $x, y, r$은 실수이고 $-10^6 \le x, y \le 10^6$, $ 0.5 \le r \le 10^7$을 만족해야 한다.
    
    답이 여러 개일 경우 그 중에서 아무거나 출력하면 된다.
   
    \Examples
    
    \begin{example}
        \exmp{
            7
            0.000000000000000 0.000000000000000
            0.000000000000000 2.000000000000000
            2.000000000000000 2.000000000000000
            2.000000000000000 0.000000000000000
            50.000000000000000 11.000000000000000
            51.000000000000000 10.000000000000000
            49.000000000000000 10.000000000000000
        }{%
            1.0 1.0 1.4142135624
            50.0 10.0 1.0
        }%
        \exmp{
            3
            -4.000000000000000 0.000000000000000
            0.000000000000000 1.012345678901234
            5.000000000000000 0.000000000000000
        }{%
            0.0 0.0 1.012345678901234
            0.5 0.0 4.50000
        }%
    \end{example}
    
    \Notes
    원의 중심과 점의 거리를 $d$라고 할 때, $d$와 반지름 $r$의 절대 혹은 상대 오차가 $10^{-3}$ 이하일 때 점은 원 위에 있다고 판단된다.
    
    $N$개의 점 모두를 서로 다른 임의의 방향으로 최대 $10^{-6}$ 만큼 움직여도 여전히 답이 존재함이 보장된다.
    
    두 원이 모두 $-9 \times 10^5 \le x, y \le 9 \times 10^5, 0.51 \le r \le 9 \times 10^6$ 조건을 만족하는 답이 존재함이 보장된다.
    
    임의의 세 점이 이루는 각이 $0.95\pi$ 이하임이 보장된다.
    
\end{problem}

