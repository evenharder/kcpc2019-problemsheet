\begin{problem}{\kcpcprobacronym\ (\kcpcprobacronymshort)}
    {표준 입력}{표준 출력}
    {\kcpcprobacronymtime\,초}{\kcpcprobacronymmemory\,MB}{}
    
    UCPC와는 다르게 KCPC는 무엇의 약자인지 이미 알려져있다. Korea University Collegiate Programming Contest가 바로 그것인데, 이미 KCPC가 무엇의 약자인지 알고 있는 상황에서 `KCPC는 무엇의 약어일까'를 내려고 한 박홍빈은 김이 팍 빠졌다. 하지만 박홍빈은 그럼에도 문제를 내고 싶었기에 문제를 약간 수정했다. 
    
    문자열의 \textbf{축약}이란 문자열에서 임의의 문자들을 제거하고 남은 문자들을 순서를 유지하며 이어붙여 새로운 문자열을 만드는 과정으로 예시는 다음과 같다.
    
    \begin{itemize}
    \item \texttt{"ABCDE"} $ \to $ \texttt{"BD"}
    \item \texttt{"Korea University Collegiate Programming Contest"} $ \to $ \texttt{"KCPC"}
    \end{itemize}
    
    어떤 문자열 $ S $를 축약해서 \texttt{"KCPC"}로 만들 수 있는지 확인하는 것은 이미 있는 문제이니 \texttt{"KCPC"}로도 주어진 문자열 $ A $로도 축약 가능한지 확인해 보기로 하였다.
    
    문제에서 요구하는 것은 문자열 $ A $와 $ S $가 주어졌을 때, 문자열 $ S $를 축약해서 문자열 $ A $를 만들 수 있는지, 또 문자열 $ S $를 축악해서 \texttt{"KCPC"}로도 만들 수 있는지 확인하는 것이다.
    
    대문자와 소문자는 다른 문자로 취급하며, 축약 과정은 독립적이다.
    
    \InputFile
    첫 번째 줄에 알파벳 대문자와 소문자로만 이루어진 문자열 $ A $가 주어진다.
    
    두 번째 줄에 알파벳 대문자와 소문자로만 이루어진 문자열 $ S $가 주어진다. 
    
    각각의 문자열 길이는 최대 1000자이다.
    
    \OutputFile
    첫 번째 줄에 문자열 $ S $를 축약하여 문자열 $ A $와 \texttt{"KCPC"}를 만들 수 있다면 \texttt{"KCPC!"}를 그렇지 않다면 \texttt{"KCPC?"}를 따옴표를 제외하고 출력한다.
    
    \Examples
    \begin{example}
        \exmp{
            LOVE
            ILOVEKCPC
        }{%
            KCPC!
        }%
        \exmp{
            KOREA
            ILOVEKCPC
        }{%
            KCPC?
        }%
        \exmp{
            CCCC
            KCPCKCPC
        }{%
            KCPC!
        }%
    \end{example}
\end{problem}

