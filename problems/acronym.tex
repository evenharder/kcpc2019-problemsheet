\begin{problem}{\kcpcprobacronym\ (\kcpcprobacronymshort)}
    {표준 입력}{표준 출력}
    {\kcpcprobacronymtime\,초}{\kcpcprobacronymmemory\,MB}{}
    
    포켓몬스터의 주인공은 지우와 피카츄이다. 피카츄는 $N$번 대회를 치룰 예정인데, 대회를 치루면 필연적으로 경험치를 획득하게 된다. 그러나 피카츄는 지우에게 항상 피카츄로 남고 싶기 때문에 라이츄로 변신하지 않길 원한다.
    
    피카츄는 경험치가 $X$ 이상이면 라이츄로 진화하며, 한 번 라이츄로 진화하면 다시 피카츄로 되돌아가지 않는다. 또 피카츄는 휴식 차원에서 매일 밤 $A$만큼 경험치가 감소하나, 0보다 작아지지는 않는다. 앞으로 피카츄가 참여할 대회의 날짜와 획득하는 경험치가 주어질 때, 대회를 모두 마치고 난 후 피카츄가 진화했는지 아닌지의 유무를 판단해보자.
    
    피카츄의 현재 경험치는 0이며, 모든 대회는 낮에 열린다.
    
    \InputFile
    첫 번째 줄에는 대회의 수 $N$, 라이츄로 진화하기 위해 필요한 경험치 $X$, 매일 밤 사라지는 경험치 $A$가 공백으로 구분되어 주어진다. $(1 \leq N \leq 100, 1 \leq X \leq 1,000, 1 \leq A \leq 100)$
    
    두 번째 줄부터 $N + 1$번째 줄까지 대회의 날짜 $d_i$와 대회 이후 획득하는 경험치 $e_i$가 공백으로 구분되어 주어진다. $(1 \leq d_i \leq 100, 1 \leq e_i \leq 100)$ 대회 날짜는 모두 다르며, 오름차순으로 주어진다.

    입력으로 들어오는 수는 모두 정수이다. 
    
    \OutputFile
    첫 번째 줄에 대회를 진행하며 피카츄가 라이츄로 진화했으면 \texttt{'Raichu'}를, 대회가 다 끝나도 피카츄로 남아있으면 \texttt{'Pikachu'}를 따옴표를 제외하고 출력한다.
    
    \Examples
    \begin{example}
        \exmp{
            3 8 3
            1 6
            2 3
            3 5
        }{%
            Raichu
        }%
        \exmp{
            1 10 2
            3 4
        }{%
            Pikachu
        }%
    \end{example}
    
    \Explanation
    첫 번째 예제에 대한 설명은 다음과 같다.
    \begin{itemize}
        \item 1일차 낮에 보유한 경험치는 6이며, 밤에 보유한 경험치는 3이다.
        \item 2일차 낮에 보유한 경험치는 6이며, 밤에 보유한 경험치는 3이다.
        \item 3일차 낮에 보유한 경험치는 8이며, 진화한다.
    \end{itemize}
    두 번째 예제에 대한 설명은 다음과 같다.
    \begin{itemize}
        \item 2일차 밤까지 보유한 경험치는 0이다.
        \item 3일차 낮에 보유한 경험치는 4이다. 이 이후로는 경험치가 감소하기만 하므로 진화하지 못한다.
    \end{itemize}
\end{problem}

