\begin{problem}{\kcpcprobpretty\ (\kcpcprobprettyshort)}
    {표준 입력}{표준 출력}
    {\kcpcprobprettytime\,초}{\kcpcprobprettymemory\,MB}{}
     
    12월 25일은 크리스마스이다. 크리스마스를 맞은 피카츄는 지우를 위해 트리를 만들기로 했다. 금전적인 문제로 살아있는 나무를 구할 수 없게 된 피카츄는 나무 대신 그래프의 일종인 트리를 꾸미기로 했다.
    
    \textbf{트리}란 모든 정점이 연결되어 있으며, 한 정점에서 다른 정점으로 가는 경로가 유일한 그래프이다. 트리에서 간선으로 연결된 두 정점 $u$와 $v$에 대해 $u$가 루트 정점에 더 가까운 정점이라고 할 때, $v$의 부모는 $u$이며 $u$는 $v$를 자식으로 가진다. 트리에 루트 정점은 단 하나 존재하며, 정의에 의해 루트 정점은 부모 정점이 없다. 정점 $x$의 \textbf{서브트리}란 $x$와 $x$의 자식들의 서브트리로 구성된 트리를 의미한다.
    
    정점 $x$에서의 \textbf{전위 순회}는 트리를 특정한 규칙에 의해 방문하는 순서로, 다음과 같이 정의된다.
    
    \begin{itemize}
        \item 정점 $x$를 방문한다.
        \item 자식 정점의 번호의 오름차순으로 (작은 번호부터 큰 번호로) 전위 순회를 한다.
    \end{itemize}
    전위 순회의 정의에 의해 전위 순회는 정점별로 유일하다. 추가적으로, 트리의 전위 순회는 루트 정점에서의 전위 순회로 정의한다.
    
    한 정점에서의 전위 순회는 그 서브트리의 모든 정점을 정확히 한 번만 방문하기 때문에 의미가 있다.
    
    수학적 정의가 귀찮았지만 트리를 그리는 건 좋았던 피카츄는 인터넷 검색을 통해 1번 정점이 루트인 트리를 그렸다. 트리를 더 장식하기 위해 각 정점에 괄호(여는 괄호 \texttt{'('} 또는 닫는 괄호 \texttt{')'})를 한 개씩 그렸는데, 트리를 본 지우는 정점이 다음의 성질을 만족하면 아름답다고 정의했다.
    
    \begin{itemize}
        \item 정점 $x$에서의 전위 순회를 하며, 방문한 순서대로 각 정점에 부여된 괄호를 추가해가며 문자열을 얻을 수 있다. 이렇게 만들 수 있는 괄호 문자열이 \textbf{올바른 괄호 문자열}이라면 $ x $는 아름답다.    
    \end{itemize}
    
    올바른 괄호 문자열은 다음과 같이 정의된다.
    \begin{enumerate}
        \item 빈 문자열은 올바른 괄호 문자열이다.
        \item \texttt{A}가 올바른 괄호 문자열이면 괄호를 씌운 \texttt{(A)}도 올바른 괄호 문자열이다.
        \item \texttt{A}와 \texttt{B}가 올바른 괄호 문자열이면 둘을 이어붙인 \texttt{AB}도 올바른 괄호 문자열이다.
    \end{enumerate}
    
    지우는 트리의 아름다움을 트리의 정점 중 아름다운 정점의 수라고 정의했다. 피카츄를 도와 피카츄가 꾸민 트리가 얼마나 아름다운지 구하자.
    
    \InputFile
    첫 번째 줄에 노드의 개수 $N(2 \leq N \leq 200,000$)이 주어진다.
    
    두 번째 줄에는 $0$과 $1$로만 이루어진 $N$개의 정수 $a_1, a_2, \cdots, a_N$가 공백으로 구분되어 주어진다. $a_i$가 0이면 정점 $i$에 여는 괄호 \texttt{'('}가 적혀 있으며, 1이면 \texttt{')'}이 적혀 있다.
    
    이후 $N-1$개 줄에 걸쳐, 두 정수 $a$와 $b$가 공백으로 구분되어 주어진다. $ (1 \leq a, b \leq N) $ 이는 정점 $a$와 정점 $b$가 간선으로 연결되어 있음을 의미한다.
    
    입력으로 들어오는 그래프는 트리이다.
    
    \OutputFile
    첫 번째 줄에 피카츄가 만든 트리의 아름다움을 출력한다.
    
    \Examples
    \begin{example}
        \exmp{
            10
            0 0 1 0 0 1 1 1 0 1
            1 2
            1 8
            1 4
            1 10
            2 5
            6 2
            2 7
            8 3
            4 9
        }{%
            2
        }%
    \end{example}
    
    \Explanation
    첫 번째 예제에서 닫는 괄호가 있는 칸에 빗금을 쳐서 그리면 다음과 같다.
    \begin{figure}[H]
    \centering
    \begin{tikzpicture}
        \begin{scope}[every node/.style={draw, minimum size=1cm}, font=\bfseries]
     			% draw nodes
        \def \h {2}
        \def \w {2}
     			\foreach \name/\xx/\yy in {
            $1$/0*\w/2*\h,
            $2$/-2*\w/1*\h,
            $4$/-0.5*\w/1*\h,
            $5$/-2.8*\w/0*\h,
            $9$/-0.5*\w/0*\h
            }{
     				\node[circle] (\name) at (\xx, \yy) {\name};
     	    }
        \end{scope} %pattern=north west lines
        \begin{scope}[every node/.style={draw, minimum size=1cm, pattern=north west lines}, font=\bfseries]
     			% draw nodes
        \def \h {2}
        \def \w {2}
     			\foreach \name/\xx/\yy in {
            $8$/0.5*\w/1*\h,
            $6$/-2*\w/0*\h,
            $7$/-1.2*\w/0*\h,
            $3$/0.5*\w/0*\h,
            $10$/1.5*\w/1*\h
            }{
     				\node[circle] (\name) at (\xx, \yy) {\name};
     	      }
        \end{scope}
        % draw edges
        %\draw (a) -- node
        \begin{scope}[font=\scriptsize]
            %\foreach \edgefrom/\edgeto/\dist in {a/b/1, b/c/3}{
      			%	\draw (\edgefrom) -- node[midway, below] {\dist} (\edgeto);
      			%}
            \foreach \edgefrom/\edgeto in {
                $1$/$2$,
                $1$/$8$,
                $1$/$4$,
                $1$/$10$,
                $2$/$5$,
                $2$/$6$,
                $2$/$7$,
                $3$/$8$,
                $4$/$9$}{
      				\draw (\edgefrom) -- (\edgeto);
      			}
        \end{scope}
    \end{tikzpicture}
    \end{figure}
    정점 1에서의 전위 순회는 1 $ \rightarrow $ \  2 $ \rightarrow $ \  5  $ \rightarrow $ \  6 $ \rightarrow $ \  7 $ \rightarrow $ \  4 $ \rightarrow $ \  9 $ \rightarrow $ \  8  $ \rightarrow $ \  3 $ \rightarrow $ \  10이고, 올바른 괄호 문자열 \texttt{((())(()))}을 생성하기에 아름다운 정점이다. 
    
    정점 2에서의 전위 순회는 2 $ \rightarrow $ \  5  $ \rightarrow $ \  6 $ \rightarrow $ \  7으로, 올바른 괄호 문자열 \texttt{(())}을 생성하기에 아름다운 정점이다.
    
    그 외에 아름다운 정점은 존재하지 않는다.
\end{problem}

