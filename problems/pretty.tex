\begin{problem}{\kcpcprobpretty\ (\kcpcprobprettyshort)}
    {표준 입력}{표준 출력}
    {\kcpcprobprettytime\,초}{\kcpcprobprettymemory\,MB}{}
    
    12월 25일은 크리스마스이다. 크리스마스를 맞은 피카츄는 지우를 위해 트리를 만들기로 했다. 금전적인 문제로 살아있는 나무를 구할 수 없게 된 피카츄는 나무 대신 그래프의 일종인 트리를 꾸미기로 했다.
    
    \textbf{트리}란 모든 정점이 연결되어 있으며, 한 정점에서 다른 정점으로 가는 경로가 유일한 그래프이다. 트리에서 간선으로 연결된 두 정점 $u$와 $v$가 있고, $u$가 루트 정점에 더 가까운 정점이라고 할 때, $v$의 부모는 $u$이며, $u$는 $v$를 자식으로 가진다. 루트 정점은 단 하나 존재하며, 정의에 의해 루트 정점은 부모 정점이 없다.
    
    정점 $x$의 \textbf{서브트리}란 루트 정점이 있는 트리에서, $x$와 $x$의 자식들의 서브트리로 구성된 트리를 의미한다. $x$를 루트로 삼되, $x$의 부모로 향하는 간선을 무시한다고 생각하면 된다.
    
    정점 $x$의 서브트리에서의 \textbf{전위 순회}는 트리를 특정한 규칙에 의해 방문하는 순서로, 다음과 같이 정의된다.
    
    \begin{itemize}
        \item 정점 $x$를 방문한다.
        \item 임의의 자식 정점의 서브트리에서 전위 순회를 한다.
        \item 모든 자식을 정확히 한 번만 방문한 후 종료한다.
    \end{itemize}
    추가적으로, 트리의 전위 순회는 루트 정점에서의 전위 순회로 정의한다.
    
    서브트리에서의 전위 순회는 그 서브트리의 모든 정점을 정확히 한 번만 방문하기 때문에 의미가 있다. 다만 정점의 자식이 많을 경우, 전위 순회가 유일하지 않을 수 있다. 
    
    수학적 정의가 귀찮았지만 트리를 그리는 건 좋았던 피카츄는 인터넷 검색을 통해 1번 정점이 루트인 트리를 그렸다. 트리를 더 장식하기 위해 각 정점에 괄호(여는 괄호 \verb|(| 또는 닫는 괄호 \verb|)|)를 한 개씩 그렸는데, 트리를 본 지우는 정점이 다음의 성질을 만족하면 아름답다고 정의했다.
    
    \begin{itemize}
        \item 정점 $ x $의 서브 트리를 전위 순회하여 얻은 괄호 문자열이 올바른 괄호 문자열이라면 $ x $는 아름답다.
        \item 전위 순회하여 얻은 괄호 문자열은 방문한 순서대로 각 정점에 부여된 괄호를 뒤에 추가하며 얻는 문자열이다.
    \end{itemize}
    올바른 괄호 문자열은 다음과 같이 정의된다.
    \begin{enumerate}
        \item 빈 문자열은 올바른 괄호 문자열이다.
        \item \verb|A|가 올바른 괄호 문자열이면 괄호를 씌운 \verb|(A)|도 올바른 괄호 문자열이다.
        \item \verb|A|와 \verb|B|가 올바른 괄호 문자열이면 둘을 이어붙인 \verb|AB|도 올바른 괄호 문자열이다.
    \end{enumerate}
    
    추가적으로, 지우는 트리의 아름다움을 트리 정점 중 아름다운 정점의 수라고 정의했다. 피카츄를 도와 피카츄가 꾸민 트리가 얼마나 아름다운지 구하자.
    
    \InputFile
    첫 번째 줄에는 노드의 개수 $N(2 \leq N \leq 10^5$)이 주어진다.
    
    두 번째 줄에는 $0$과 $1$로만 이루어진 $N$개의 정수가 주어지는데, $i$번째 수가 $0$이면 여는 괄호 $1$이면 닫는 괄호를 의미한다.
    
    이후 $N-1$줄에 걸쳐, $a_i$번 노드와 $b_i$번 노드를 잇는 간선의 정보가 주어진다.
    
    입력으로 들어오는 그래프는 트리임이 보장된다.
    
    \OutputFile
    첫 번째 줄에 피카츄가 만든 트리의 아름다움을 출력하자.
    
    \Examples
    \begin{example}
        \exmp{
            10
            0 0 1 0 0 1 1 1 0 1
            1 2
            1 3
            1 4
            1 10
            2 5
            2 6
            2 7
            3 8
            4 9
        }{%
            2
        }%
    \end{example}
\end{problem}

