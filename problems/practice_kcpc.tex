\begin{problem}{\kcpcpprobkcpc\ (\kcpcpprobkcpcshort)}
    {표준 입력}{표준 출력}
    {\kcpcpprobkcpctime\,초}{\kcpcpprobkcpcmemory\,MB}{}
    
    고려대학교 프로그래밍 경시대회(KCPC)에 참가하게 되신 걸 진심으로 축하드립니다!
    
    \texttt{"Hello world!"}를 처음 출력했던 마음으로 \texttt{"Hello, KCPC!!"}를 따옴표를 제외하고 출력해봅시다.
    
    \InputFile
    입력은 문자열 \texttt{"Hello world!"}이다. 따옴표는 경계를 나타내기 위해 있을 뿐, 문자열의 일부가 아니다.
    
    \OutputFile
    첫 번째 줄에 \texttt{"Hello, KCPC!!"}를 따옴표를 제외하고 출력한다.
   
    \Examples
    
    \begin{example}
        \exmp{
            Hello world!
        }{%
            Hello, KCPC!!
        }%
    \end{example}
    \Note
    온라인 저지의 채점 방식 상, 이 문제는 입력을 받지 않고 \texttt{"Hello, KCPC!!"}만 출력해도 된다.
\end{problem}

