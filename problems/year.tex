\begin{problem}{\kcpcyeartitle}
    {표준 입력}{표준 출력}
    {\kcpcyeartime\,초}{\kcpcyearmemory\,MB}{}{\kcpcyearscore}
    
    조그만 수학적 연관성에도 깊은 흥미를 두는 상헌이는 우연히 고려대학교 프로그래밍 경시대회가 열리는 날짜를 년도와 월일을 붙여 쓰면 20181208임을 알게 되었다. 2018년이 한 달도 남지 않음을 깨달은 상헌이는 수학적 감수성에 휩싸여, 이 수가 숫자 2, 0, 1, 8로만 이루어져 있는 사실에 심취하였다. 상헌이는 다사다난했던 2018년을 추억하기 위해 2, 0, 1, 8로만 이루어져 있는 정수를 생각하기 시작하였고, 그 결과 상헌이는 양의 정수를 다음과 같이 4종류로 나누어 정의하였다.
    
    어떤 양의 정수를 10진수로 나타냈을 때 2, 0, 1, 8로만 이루어져 있다면 이는 2018과 \textbf{관련 있는} 수라고 부른다. 2018과 \textbf{관련 있는} 수 중에서 2, 0, 1, 8을 모두 포함하는 수들은 2018과 \textbf{밀접한} 수이다. 2018과 \textbf{밀접한} 수 중에서 2, 0, 1, 8의 개수가 모두 똑같은 수들은 2018과 \textbf{묶여있는} 수이다. 2018과 연관된 수가 아닌 양의 정수는 2018과 \textbf{관련 없는} 수이다.
    
    상헌이를 도와 어떤 양의 정수가 2018과 어느 정도의 관련이 있는지 파악해보자.
    
    \InputFile
    첫 번째 줄에 양의 정수 $ N $이 주어진다. $ (1 \leq N < 1,000,000,000) $
    
    $ N $은 0으로 시작하지 않는다.
    
    \OutputFile
    첫 번째 줄에 $ N $이 2018과 \textbf{관련 없는} 수이면 0을,
    
    2018과 \textbf{관련 있는} 수이나 \textbf{밀접한} 수가 아니면 1을,
    
    2018과 \textbf{밀접한} 수이나 \textbf{묶여있는} 수가 아니면 2를,
    
    2018과 \textbf{묶여있는} 수이면 8을 출력한다.
    
    \Examples
    
    \begin{example}
        \exmp{
            20181208
        }{%
            8
        }%
        \exmp{
            1208021
        }{%
            2
        }%
        \exmp{
            10
        }{%
            1
        }%
        \exmp{
            4
        }{%
            0
        }%
    \end{example}
    
\end{problem}

