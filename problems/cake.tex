\begin{problem}{\kcpcprobcake\ (\kcpcprobcakeshort)}
    {표준 입력}{표준 출력}
    {\kcpcprobcaketime\,초}{\kcpcprobcakememory\,MB}{}
    
    2019년도 어언 한 달밖에 남지 않았다. 다사다난했던 2019년을 기념하기 위하여 상헌이는 고려대학교 프로그래밍 경진대회 운영진과 출제진과 나누어먹을 케이크를 장식하고자 한다.
    
    균형과 일관성을 극도로 중시하는 상헌이는 정사각형 케이크를 샀고, 이를 가로 $n$줄 세로 $n$줄이 되게 조각으로 만들었다. 미니멀리즘이 듬뿍 담겨있는 이 케이크 위에는 아무 장식도 뿌려져있지 않다. 너무 밋밋하다고 생각한 상헌이는 케이크에 초코칩을 다음과 같이 여러 번 뿌려 장식을 하려고 한다.
    
    \begin{itemize}
        \item 한 가로줄을 선택하여, 이 가로줄에 속한 모든 조각에 초코칩을 1개 추가한다.
        \item 한 세로줄을 선택하여, 이 세로줄에 속한 모든 조각에 초코칩을 1개 추가한다.
    \end{itemize}
    
    상헌이는 가장 초코칩이 많이 뿌려져 있는 조각을 `가장 맛있는 조각'이라고 부른다.
    
    매번 장식을 하는 상헌이는 가장 맛있는 조각의 개수를 신경쓰지 않을 수 없다. 다행히 상헌이는 프로그램을 작성하여 자신의 행동에 따른 가장 맛있는 조각의 개수를 성공적으로 계산해내었고, 이를 다음과 같이 문제로 만들었다. 케이크에 장식을 올리는 느낌으로 문제를 풀어보자.

    
    \InputFile
    첫 번째 줄에 두 정수 $n$, $q$가 공백으로 구분되어 주어진다. $(1 \leq n, q \leq 30,000)$
    
    $n$은 가로줄과 세로줄의 개수이며, $q$는 장식을 하는 횟수이다. 
    
    이후로 $q$개의 줄에 두 정수 $t$, $a$가 공백으로 구분되어 주어진다. ($1 \leq t \leq 2,\ 1 \leq a \leq n$)
    
    $t$가 1이면 $a$번째 가로줄에, 2이면 $a$번째 세로줄에 있는 조각들에 초코칩을 하나씩 더한다.
    
    \OutputFile
    각 줄마다 매 번 장식을 한 이후 그 상태의 가장 맛있는 조각의 개수를 출력한다.
    
    모든 초코칩은 장식이 끝난 후에도 유지된다.
    
    \Examples
    \begin{example}
        \exmp{
            3 2
            1 1
            1 3
        }{%
            3
            6
        }%
        \exmp{
            1 3
            1 1
            2 1
            1 1
        }{%
            1
            1
            1
        }%
        \exmp{
            4 5
            1 1
            1 4
            2 3
            1 4
            2 2
        }{%
            4
            8
            2
            1
            2
        }%
    \end{example}
\end{problem}

