\begin{problem}{\kcpcprobfirst\ (\kcpcprobfirstshort)}
    {표준 입력}{표준 출력}
    {\kcpcprobfirsttime\,초}{\kcpcprobfirstmemory\,MB}{}
    
    매년 이 시기가 되면 새내기는 헌내기가 되는 것에 두려움을 느낀다. 심지어 2020년 새내기는 2001년 생이다.
    
    이것에 충격받은 16학번 화석 유신이는 다시 새내기가 되고자 대학교를 재입학하기로 마음먹었다.
    유신이는 평생 새내기가 되고 싶지만, 고학년도 되고 싶은 마음이 있기에 다음과 같은 규칙을 가지고 학교 생활을 하기로 했다.
    
    \begin{itemize}
        \item 어떤 년도에 (재)입학한 사람의 학번은 그 년도의 마지막 2자리이며, 학년은 1학년이다.
        \item 한 해가 지나면 학년이 하나 증가한다. 단, 4학년이 된 해의 다음 해는 새내기(1학년)로 재입학을 한다.
    \end{itemize}
    
    예를 들면 유신이의 앞으로의 일정은 다음과 같다.
    
    \begin{itemize}
        \item 유신이는 2020년에 재입학하면 20학번 1학년이 된다.
        \item 2021년에는 20학번 2학년이 된다.
        \item 2022년에는 20학번 3학년이 된다.
        \item 2023년에는 20학번 4학년이 된다.
        \item 4학년이 된 유신이는 다음해에 다시 재입학한다.
        \item 즉 유신이는 2024년에 24학번 1학년이 된다.
    \end{itemize}
    
    유신이는 금수저기 때문에 이 과정으로 노년까지 하고자 한다.
    유신이는 $X$년에 몇 학번, 몇 학년일까?
    
    단, 유신이는 2016년도에 1학년이라고 가정한다.
    
    
    \InputFile
    첫 번째 줄에는 유신이의 학번이 궁금한 년도 $X$가 입력으로 들어온다. $(2016 \leq X \leq 2100)$
    
    \OutputFile
    첫 번째 줄에 유신이의 학번 두 자리와 학년 한 자리를 공백으로 구분하여 출력한다.
    
    \Examples
    \begin{example}
        \exmp{
            2016
        }{%
            16 1
        }%
        \exmp{
            2019
        }{%
            19 4
        }%
        \exmp{
            2100
        }{%
            00 1
        }%
    \end{example}

\end{problem}

