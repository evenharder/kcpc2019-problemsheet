\begin{problem}{\kcpcprobserver\ (\kcpcprobservershort)}
    {표준 입력}{표준 출력}
    {\kcpcprobservertime\,초}{\kcpcprobservermemory\,MB}{}
    
    ``지난 밤, 선량한 서버가 죽었습니다. 관리자들은 모두 고개를 들어주세요."
    
    서버 관리자에게 서버가 다운되는 것만큼 무섭고 고된 일은 없을 것이다. 상헌이도 그 마수에서 벗어날 수 없었다. 상헌이는 자신이 도맡아 관리하는 서버들이 뻗는 걸 원하지 않기에 서버 용량을 늘리는 방향을 선택했다.
    
    상헌이는 회사로부터 $2^0$, $2^1$, ..., $2^{k-1}$ MB의 용량을 지니고 있는 서로 다른 디스크를 각 용량별로 $a$개씩 지원받았다. 딥러닝을 열심히 돌려본 결과 서버에 정확히 $n$ MB의 용량을 추가로 늘리면 최적의 성능을 발휘할 것으로 분석되었다. 문제는, $n$ MB의 용량을 만드는 방법이 너무 다양했다는 것이다! 상헌이는 서버 증축을 할 수 있는 방법의 수를 구해보고자 한다. 그 수가 너무 커질 걸 우려해, $1048573$으로 나눈 나머지를 계산하려고 한다. 
    $1048573 = 2^{20} - 3$은 소수이다.
    
    \InputFile
    첫 번째 줄에는 세 정수 $k$, $a$, $n$이 주어진다. ($1 \leq k \leq 40, 1 \leq a \leq 10^6, 1 \leq n \leq 10^{15}$)
    
    $k$는 제공되는 디스크의 용량의 가짓수, $a$는 각 용량별 서로 다른 디스크의 개수, $n$은 증축해야 하는 용량을 의미한다.
    
    \OutputFile
    첫 번째 줄에 증축할 수 있도록 디스크를 선택하는 경우의 수를 $1048573$으로 나눈 나머지를 출력한다.
    
    \Examples
    \begin{example}
        \exmp{
            2 3 5
        }{%
            12
        }%
        \exmp{
            5 10 1000
        }{%
            0
        }%
        \exmp{
            11 23 58
        }{%
            182185
        }%
    \end{example}
    
    \Note
    이 문제를 풀 때 뤼카의 정리(Lucas' Theorem)을 사용하면 편리할 것이다. 
    
    뤼카의 정리란, 양의 정수 $m$, $n$이 소수 $p$에 대해 $p$진법으로 각각 $m_{k}m_{k-1}\cdots m_{0}{_{(p)}}$, $n_{k}n_{k-1}\cdots n_{0}{_{(p)}}$으로 표현될 때, $$\binom{m}{n} \equiv \prod_{i=0}^{k} \binom{m_i}{n_i} \pmod p$$이 성립한다는 것이다. $\binom{n}{r} = \frac{n!}{r!(n-r)!}$이며, $n < r$일 때 $\binom{n}{r} = 0$으로 정의한다.
    
\end{problem}

