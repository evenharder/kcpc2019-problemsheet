\begin{problem}{\kcpcpprobnotfibo\ (\kcpcpprobnotfiboshort)}
    {표준 입력}{표준 출력}
    {\kcpcpprobnotfibotime\,초}{\kcpcpprobnotfibomemory\,MB}{}
    
    $ a_1 = 1 $, $ a_2 = 1 $이라 잡고 $ a_n = a_{n-1} + a_{n-2}\ (n \geq 3) $으로 정의되는 수열 $ \{a_k\} $ 를 피보나치 수열이라 한다. 처음 10개의 항을 나열하면 1, 1, 2, 3, 5, 8, 13, 21, 34, 55이다.
    
    피보나치 수열을 나열하는 것은 쉽다. 그럼 피보나치 수열에 포함되지 않는 양의 정수들을 작은 순서대로 나열하는 건 어떨까? 처음 10개의 항을 나열하면 4, 6, 7, 9, 10, 11, 12, 14, 15, 16이다. 피보나치 수열에 없는 수 중 $ n $번째로 작은 수를 구해보자.
    
    \InputFile
    
    첫 번째 줄에는 양의 정수 $ n $이 주어진다. ($ 1 \leq n \leq 1,000,000 $)
    
    \OutputFile
    
    첫 번째 줄에 위에서 정의된 피보나치 수열에 포함되지 않는 양의 정수 중 $ n $번째로 작은 수를 출력한다.
    
    \Examples
    \begin{example}
        \exmp{
            1
        }{%
            4
        }%
        \exmp{
            4
        }{%
            9
        }%
        \exmp{
            12
        }{%
            18
        }%
    \end{example}
    
\end{problem}

